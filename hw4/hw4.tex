\documentclass{article}[12pt]

%% Adam Shaw, April 2015
%% in conjunction with CMSC 23500, Spring 2015, Univ of Chicago

\setlength{\parskip}{8pt}
\setlength{\parindent}{0cm}

\newcommand{\assignment}{Homework 4}
\newcommand{\whoami}{Melissa Grossbarth}
\newcommand{\assignmentdate}{May 11, 2015}

% various macros for relational algebra...
% edit, and supplement with your own

% rel : relation
\newcommand{\rel}[1]{\ensuremath{#1}}

% conj : conjunction
\newcommand{\conj}[2]{\ensuremath{#1 \wedge #2}}

% disj : disjunction
\newcommand{\disj}[2]{\ensuremath{#1 \vee #2}}

% nj : natural join 
\newcommand{\nj}[2]{\ensuremath{#1 \bowtie #2}}

% sel : select
\newcommand{\sel}[2]{\ensuremath{\sigma_{#1} #2}}

% proj : project
\newcommand{\proj}[2]{\ensuremath{\pi_{#1} #2}}

% ren : rename
\newcommand{\ren}[2]{\ensuremath{\rho_{#1} #2}}

% ...

\begin{document}

{\large \bf CMSC 23500: Introduction to Databases}

\assignment \\
\whoami \\
\assignmentdate

\hrulefill

% a few expressions to get you started...

\begin{enumerate}

\item[]Let \rel{f} represent flights and \rel{a} represent airports.

\item Let A1 = \proj{dst}({\sel{src="XYZ"}(\rel{f})})

\item[] Then the answer is N1 = \proj{name}(\nj{\rel{a}}{A1})

\item Let A2 = \proj{dst}(\nj{\rel{f}}({\ren{dst \rightarrow src} (A1)}))

\item[] Then the answer is N2 = \proj{name}(\nj{\rel{a}}{A2})

\item Let A3 = \proj{dst}(\nj{\rel{f}}({\ren{dst \rightarrow src} (A2)}))

\item[] Then the answer is N3 = \proj{name}(\nj{\rel{a}}{A3})

\item \disj{N1}{\disj{N2}{N3}}

\end{enumerate}

\end{document}